\documentclass[10pt, letterpaper]{article}

%--- Packages ---
\usepackage[utf8]{inputenc}
\usepackage[T1]{fontenc}
\usepackage{geometry}
\usepackage{hyperref}
\usepackage{enumitem}
\usepackage{xcolor}
\usepackage{tabularx}

%--- Page Layout ---
\geometry{letterpaper, left=0.5in, right=0.5in, top=0.4in, bottom=0.4in}
\pagestyle{empty}

%--- Colors & Links ---
\definecolor{linkcolor}{rgb}{0.0, 0.0, 0.8}
\definecolor{sectioncolor}{rgb}{0.2, 0.2, 0.2}
\hypersetup{
    colorlinks=true,
    urlcolor=linkcolor,
    linkcolor=linkcolor,
    pdftitle={Wenpeng Xu Resume},
    pdfauthor={Wenpeng Xu},
}

%--- Section Title Command ---
\newcommand{\sectiontitle}[1]{%
  \vspace{2pt}
  {\color{sectioncolor}\Large\bfseries #1}
  \vspace{-5pt}
  \rule{\linewidth}{0.6pt}
  \vspace{1pt}
}

%--- List Spacing ---
\setlist[itemize]{topsep=0pt, partopsep=0pt, parsep=0pt, itemsep=0.5pt, leftmargin=*}

% =========================================================
\begin{document}
% =========================================================

%--- Header ---
\begin{center}
  {\huge \textbf{Wenpeng Xu}}
  \vspace{2pt}

  (858) 241-4024 \quad$\cdot$\quad
  \href{mailto:wex019@ucsd.edu}{wex019@ucsd.edu} \quad$\cdot$\quad
  \href{https://www.linkedin.com/in/wenpeng-xu-725786387}{linkedin.com/in/wenpeng-xu} \quad$\cdot$\quad
  \href{https://github.com/OwenXu6}{github.com/OwenXu6}
\end{center}

\vspace{-6pt}

% ---------------------------------------------------------
\sectiontitle{Education}
% ---------------------------------------------------------

\textbf{University of California, San Diego (UCSD)} \hfill San Diego, CA \\
\textit{M.S. in Computer Engineering, Specialization: AI \& Hardware Design} \hfill Sep 2025 -- Present
\vspace{1pt}

\textbf{Tongji University} \hfill Shanghai, China \\
\textit{B.E. in Software Engineering} \hfill Sep 2020 -- Jun 2025 \\
\textit{GPA: 3.80 / 4.0}

% ---------------------------------------------------------
\vspace{-2pt}
\sectiontitle{Skills}
% ---------------------------------------------------------

\begin{itemize}
  \item \textbf{Hardware Design}: Verilog, FPGA, ASIC Flow, ModelSim, VCS, DC Synthesis,
        Cadence Innovus P\&R, TSMC 65nm PDK
  \item \textbf{AI \& Algorithms}: PyTorch (QAT), Transformer, LLM API (Tool Use / Agentic Loop),
        Signal Processing (STFT/FFT), Unsupervised Learning
  \item \textbf{Programming Languages}: Python, C++, Java, C\#, Assembly, SQL
  \item \textbf{Core Concepts}: HW-SW Co-design, SIMD Architecture, Low-Power Design (Clock Gating),
        Memory Optimization, Distributed Caching, SSH Automation
\end{itemize}

% ---------------------------------------------------------
\vspace{-2pt}
\sectiontitle{Experience}
% ---------------------------------------------------------

\textbf{China Resources Digital Technology} \hfill Shenzhen, China \\
\textit{AI Engineer Intern --- Industrial IoT \& Signal Processing} \hfill Sep 2024 -- Dec 2024
\begin{itemize}
  \item Developed an \textbf{unsupervised anomaly detection system} for industrial machinery
        (cement manufacturing equipment), leveraging vibration sensor data to predict equipment failures.
  \item Applied \textbf{Short-Time Fourier Transform (STFT)} and power spectral analysis on large-scale
        time-series data, converting raw 1-D signals into spectral feature maps for deep learning.
  \item Built a \textbf{Transformer-based autoencoder} pipeline that captures long-range temporal
        dependencies in sensor logs, improving detection accuracy by \textbf{15\%} over traditional
        autoencoders.
  \item Benchmarked semi-supervised approaches (KNN, OCSVM) to enable adaptation to novel anomaly
        patterns with minimal human annotation.
\end{itemize}

\vspace{3pt}

\textbf{Alibaba Cloud} \hfill Beijing, China \\
\textit{Software Development Engineer Intern} \hfill Jun 2025 -- Aug 2025
\begin{itemize}
  \item Designed a \textbf{Redis-based caching layer} for a large-scale management platform to optimize
        core search functionality.
  \item Implemented a hybrid cache eviction strategy (proactive + TTL-based deletion) to ensure data
        consistency under strict memory constraints, eliminating cache penetration and stale-read risks.
  \item Reduced database I/O latency and achieved a \textbf{60\% improvement} in search operation
        throughput, significantly lowering P99 latency under high concurrency.
\end{itemize}

% ---------------------------------------------------------
\vspace{-2pt}
\sectiontitle{Projects}
% ---------------------------------------------------------

\textbf{IC Design Automation AI Agent} \hfill Jan 2026 -- Mar 2026
\begin{itemize}
  \item Built a multi-step \textbf{autonomous AI Agent} using \textbf{Claude API (Streaming + Tool Use)}
        that drives a complete chip design flow (RTL $\rightarrow$ EDA Synthesis $\rightarrow$ Physical
        P\&R) end-to-end without human intervention.
  \item Designed \textbf{7 modular tools} (file I/O, local/SSH remote execution, one-shot project sync)
        and a \textbf{ReAct-style loop} enabling the agent to autonomously invoke tools, parse outputs,
        and iteratively fix errors.
  \item Implemented exponential-backoff \textbf{rate-limit retry} and a configurable-timeout SSH
        execution engine (Python + Paramiko), achieving 100\% task success rate for 10+ minute jobs.
  \item Successfully taped out two TSMC 65nm GP designs --- 4-bit adder and 8-bit ALU @ 200\,MHz ---
        with timing closure (WNS $>$ 0) and zero DRC violations.
\end{itemize}

\vspace{3pt}

\textbf{High-Efficiency Reconfigurable AI Accelerator Based on VGG16}
\hfill Oct 2025 -- Dec 2025 \\
{\small\textit{Top Performer / High Distinction --- ECE 284 Parallel Processing, UCSD}}
\begin{itemize}
  \item Designed a deep learning edge accelerator in \textbf{Verilog} supporting \textbf{SIMD} and
        reconfigurable architecture for mixed-precision (2-bit/4-bit) inference.
  \item Performed \textbf{HW-SW co-design}: applied \textbf{Quantization-Aware Training (QAT)} to
        VGG16 in PyTorch to minimize accuracy loss, then verified hardware logic against the
        software Golden Model.
  \item Integrated \textbf{Clock Gating} and weight-fusion techniques to reduce dynamic power and
        maximize memory bandwidth utilization, improving overall Performance/Watt.
  \item Achieved \textbf{100\% functional verification coverage} via ModelSim/VCS automated testbenches
        and completed synthesis with optimal PPA trade-offs.
\end{itemize}

\vspace{3pt}

\textbf{Multi-task Transformer for Joint Accent \& Gender Classification}
\hfill Jul 2023 -- Aug 2023 \\
{\small\textit{Published --- EI International Conference}}
\begin{itemize}
  \item Developed a \textbf{multi-task Transformer} model for joint classification of speaker accent
        and gender from audio recordings.
  \item Leveraged \textbf{Wav2Vec} embeddings combined with \textbf{CTC Loss} and cross-entropy to
        enable efficient multi-objective optimization in a shared encoder framework.
  \item Preprocessed and curated \textbf{2,484+ hours} of audio data from the Common Voice dataset
        for large-scale model training and evaluation.
\end{itemize}

\end{document}
